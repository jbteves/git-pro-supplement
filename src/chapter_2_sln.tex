\section{Getting a Git Repository}

\subsection{Questions for Understanding}
\begin{enumerate}
    \item Turning a local directory into a git repository, or cloning an
        existing repository to your local machine.
    \item The full copy of all files and changes.
\end{enumerate}

\subsection{Exercises}
\begin{enumerate}
    \item Creating the folder may vary, but using bash:
        \begin{verbatim}
            mkdir recipes
            cd recipes/
            git init
        \end{verbatim}
        You should see a message from git that reads something like this:
        \begin{verbatim}
            Initialized empty Git repository in /Users/person/Downloads/recipes/.git/
        \end{verbatim}
    \item \verb+git clone https://github.com/jbteves/git-pro-supplement.git+
        You will be prompted for a GitHub username and password.
        After authenticating, you should see a message from git that looks
        like this:
        \begin{verbatim}
            remote: Enumerating objects: 19, done.
            remote: Counting objects: 100% (19/19), done.
            remote: Compressing objects: 100% (13/13), done.
            remote: Total 19 (delta 6), reused 15 (delta 6), pack-reused 0
            Receiving objects: 100% (19/19), 7.46 KiB | 1.07 MiB/s, done.
            Resolving deltas: 100% (6/6), done.
        \end{verbatim}
        The number of objects, deltas, and other values will be different
        since this repository will evolve in the time since I've run the
        command.
\end{enumerate}
