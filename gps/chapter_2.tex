\section{Getting a Git Repository}

\subsection{Questions for Understanding}
\begin{enumerate}
    \item What are the two standard ways of obtaining a git repository?
    \item What do you receive when you clone a repository by default?
\end{enumerate}

\subsection{Exercises}
\begin{enumerate}
    \item Create a directory, \verb+recipes+, and initialize it as a git
        repository.
    \item Clone this git repository from the url below
        \url{https://github.com/jbteves/git-pro-supplement.git}
\end{enumerate}

\section{Recording Changes to the Repository}

\subsection{Questions for Understanding}
\begin{enumerate}
    \item If you add a file to an empty repository, is it tracked or
        untracked?
    \item Can you add folders with the \verb+git add+ command?
    \item When running \verb+git status+, you can see that a file has been
        modified with changes to be committed, \emph{and} modified with
        changes not staged for commit. What happened?
    \item When running \verb+git status -s+, what is the left column? The
        right?
    \item What expression would you use to ignore all \verb+.pdf+ files in a
        repository in the \verb+.gitignore+?
    \item What's the difference between
        \verb+git diff+ and \verb+git diff --cached+?
    \item If you use \verb+git commit+ with no arguments, when does the
        commit get created?
    \item If you modify a file or stage it, and then attempt to remove it
        with \verb+git rm+,which flag must you add?
    \item To stop tracking a file but retain it in the working area, what
        flag must you add to \verb+git rm+?
    \item Does git track whether files are renamed?
\end{enumerate}

\subsection{Exercises}
Throughout the exercises, it is recommended that you routinely run
\verb+git status+ in order to see how the repository views your changes.
Exercises should be done in order.
\begin{enumerate}
    \item Go to your recipes repository, which you made in the last chapter.
        Create a file called \verb+README.txt+, and put a brief message in
        it, perhaps, ``I like cookies.'' Stage, and then commit, the file.
    \item Create a file, \verb+cookies.txt+, and add some ingredients to it.
        Stage the changes.
        Then, go back to the file and add cooking instructions.
        Commit your changes, stage the remaining ones, and then commit them.
        This exercise is to reinforce the nature of the working and staging
        areas.
    \item Create a file, \verb+ignoreme.txt+, and set up the repository to
        ignore it.
        (Hint: you will need to commit changes to the \verb+.gitignore+).
        Commit the changes to the \verb+.gitignore+.
        Check the working area and \verb+git status+ to verify that the file
        is present, but ignored by the repository.
    \item Set up the repository to ignore all \verb+.pdf+ and \verb+.docx+ 
        files.
    \item Create a file, \verb+hello.txt+, with ``Hello, world!'' in it,
        and then commit it. Stop tracking it without deleting the file from
        disk.
        What does the status say after you do so?
        After checking, unstage the change.
    \item Change \verb+hello.txt+ to \verb+hello_world.txt+ without using 
        \verb+git mv+.
        Then, change it back using \verb+git mv+.
\end{enumerate}

\section{Viewing the Commit History}

\subsection{Questions for Understanding}
\begin{enumerate}
    \item Which specifier for \verb+pretty+ gives the author email?
    \item Why are there so many options for showing the git log?
\end{enumerate}


\subsection{Exercises}
Use the \verb+recipes+ repository for this, even though there's not much
there.
\begin{enumerate}
    \item In \verb+recipes+, view the patches of the latest four commits.
    \item View all commits since last week.
    \item View all commits since the first of the month, using the ISO
        datetime format (YYYY-MM-DD).
    \item View all commits, with a format showing only:
        \begin{enumerate}
            \item the short hash
            \item the author name
            \item the commit subject
        \end{enumerate}
    \item View the last three commits, with a format showing only:
        \begin{enumerate}
            \item short hash
            \item commit subject
        \end{enumerate}
\end{enumerate}
